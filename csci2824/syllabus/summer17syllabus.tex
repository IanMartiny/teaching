% This syllabus template was created by:
% Brian R. Hall
% Assistant Professor, Champlain College
% www.brianrhall.net

% Document settings
\documentclass[11pt]{article}
\usepackage[margin=1in]{geometry}
\usepackage[pdftex]{graphicx}
\usepackage{multirow}
\usepackage{setspace}
\pagestyle{plain}
\setlength\parindent{0pt}

\begin{document}

% Course information
\begin{tabular}{ l l }
  \multirow{3}{*}{\includegraphics[height=1.25in,width=1in]{cuboulder-large.png}} & \LARGE CSCI 2824-300\\\\
  & \LARGE Discrete Structures\\\\
  & \LARGE MTW: 5pm-6:40pm, FLMG 156 \\\\
\end{tabular}
\vspace{10mm}

% Professor information
\begin{tabular}{ l l }
  % \multirow{6}{*}{\includegraphics[height=1.25in,width=1in]{vacationpic}} 
  & \large Ian Martiny\\\\
  & \large \texttt{ian.martiny@colorado.edu}\\
  & \large Moodle - \texttt{moodle.cs.colorado.edu}, password: \texttt{IanDiscreteStructures}\\
  & \large Office Hours: TR: 2pm-3pm, CSEL\\
\end{tabular}

\vspace{5mm}
This syllabus represents a rough guide of what should be expected from this course. Everything is subject to change as necessary. \\


% Course details
\textbf {\large \\ Course Description:} Covers foundational materials for computer science that is often assumed in advanced courses. Topics include set theory, Boolean algebra, functions and relations, graphs, propositional and predicate calculus, proofs, mathematical induction, recurrence relations, combinatorics, discrete probability. Focuses on examples based on diverse applications of computer science. 

Basically---math, I'll be teaching you math.\\
\textbf {Prerequisite(s):} Data-Structures (CSCI 2270) and \emph{the ability to program {\bf well} is some programming language}, preferably \texttt{C++} or \texttt{Python}.

\textbf {Credit Hours:} 3 \\

\textbf{\large Optional Books:}
\begin{itemize}
  \item Discrete Mathematics and Its Applications---Kenneth Rosen

  \item Concrete Mathematics---Ronald Graham, Donald Knuth, Oren Patashnik
\end{itemize}

\textbf {\large Why Discrete Structures:} \\
Computer science is all about solving problems, specifically with computers. This course will provide the background information necessary for computer scientists to talk intelligibly about the area.
Specifically we will garner the knowledge necessary to answer \emph{real} problems from topics such as:
\begin{enumerate} \itemsep-0.4em
  \item Logic---Propositional and First order logic, Boolean algebras.
  \item Proofs---Primer on writing proofs.
  \item Sets, Relations, and Functions---Basic properties, paradoxes! Infinite sets.
  \item Recursion---Recursive functions and recursively defined structures.
  \item Combinatorics---Counting, binomial theorem.
  \item Trees---Definition and properties 
  \item Graphs---Definition and properties
\end{enumerate}

% I recommend using \newpage here if necessary
\textbf {\large Grade Distribution:} \\
\hspace*{40mm}
\begin{tabular}{ l l }
Assignments & 35\% \\
Programs & 25\% \\
Quizzes  & 15\% \\
Final Exam  & 25\%
\end{tabular} \\\\

% Course Policies. These are just examples, modify to your liking.
\textbf {\large Course Work:}
\begin{itemize}
	\item \textbf {General}
          This is a summer course, and as such we have very little time to do a
          lot of work. You should expect to be doing a lot of work everyday for
          this course.
	\item \textbf {Grades}
		\begin{itemize}
			\item Grades in the \textbf{C} range represent performance that 
            \textbf{meets expectations}; Grades in the \textbf{B} range
            represent performance that is \textbf{substantially better} than the
            expectations; Grades in the \textbf{A} range represent work that is 
            \textbf{excellent}.
		\end{itemize}
	\item \textbf {Labs and Assignments}
		\begin{itemize}
          \item There will be one homework assignment every week. You are
          expected to work through these assignments and understand the material
          that is being tested by them.
          \item There will also be at least 2 programming assignments in a
          language of your choice.
          \item \textbf{No late assignments will be accepted under any
          circumstances}.
          \item There will be many opportunities for extra credit throughout the
          course. In particular homework that is written up in \LaTeX\ will
          receive extra points added to their total. Additionally, for
          programming assignments, code written in a language not taught by the
          department (\texttt{C/C++, Python}) will receive extra points.
		\end{itemize}
        \item You are expected to solve all problems and write all programs
        yourself. Working with others in the class is encouraged in order to
        find solutions
        but everything that is submitted must be your own. \textbf{You} must
        write your homework up and program your programs. Passing other's work
        off as your is plagiarism.
\end{itemize}

% College Policies

\textbf{Instructor's Intended Purpose}

\hspace{3mm}
\hangindent=5mm The student's work must match the instructor's intended purpose
for an assignment. While the instructor will establish the intent of an
assignment, each student must clarify outstanding questions of that intent for a
given assignment. 

\textbf{Unauthorized/Excessive Assistance}

\hspace{3mm}
\hangindent=5mm The student may not give or get any unauthorized or excessive
assistance in the preparation of any work.

\textbf{Authorship}

\hspace{3mm}
\hangindent=5mm The student must clearly establish authorship of a work.
Referenced work must be clearly documented, cited, and attributed, regardless of
media or distribution. 

\textbf{Declaration}

\hspace{3mm}
\hangindent=5mm Online submission of, or placing one's name on an exam,
assignment, or any course document is a statement of academic honor that the
student has not received or given inappropriate assistance in completing it and
that the student has complied with the Academic Honesty Policy in that work.
\end{document}



